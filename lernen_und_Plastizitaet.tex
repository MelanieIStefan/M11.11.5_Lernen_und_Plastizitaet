\documentclass{beamer}	
\mode<presentation>
 
\usepackage{pdfpages}
\usepackage{fancyvrb}
\usepackage{chemarr}

\usepackage{amsmath}		%% mathematics typesetting
\usepackage{amssymb}
 
\usepackage{epigraph}   %% nice setting of quotations

\usepackage{tabularx} %% allows to use row colours in tables

\usepackage{ulem}

\usepackage{booktabs}

\usepackage{siunitx} %% tpyeset SI units

\usepackage{CJKutf8} %% typeset Chinese characters

\usepackage{pdfpages}%% include pdfs

\usepackage{graphicx}
\usepackage{animate} %% show animated gifs

\DeclareMathAlphabet{\mathcalligra}{T1}{calligra}{m}{n}


% Color and Theme. Can be changed. However, this one's quite nice.
\usetheme{Madrid}
\definecolor{theme}{rgb}{0.84,0,0.21}
\usecolortheme[named=theme]{structure}


%%  Title information
\title[M11.11.5 Lernen, Gdächtnis]{M11.11.5 Zentrales Nervensystem 3: \\ Lernen und Gedächtnis}
\author[melanie.stefan@medicalschool-berlin.de]{}
\institute[]{Prof. Melanie Stefan \\ melanie.stefan@medcialschool-berlin.de}
\date{SoSe 2022}
 

% Table of contents to pop up at the beginning of each section
\AtBeginSection[]
{
  \begin{frame}<beamer>
    \frametitle{Outline}
    \tableofcontents[currentsection,currentsubsection]
  \end{frame}
}
 
\beamertemplatenavigationsymbolsempty

\begin{document}


{ \usebackgroundtemplate{\includegraphics[width=1.2\paperwidth]{MSB_Titelseite.pdf}} 
\begin{frame}

 \maketitle 

$\,$\\[6cm] 


\end{frame} 
}


%% Hook + TLIA

%% Water maze?
\begin{frame}
  \frametitle{Was passiert hier?}

\begin{center}
\includegraphics[width=0.6\textwidth]{MorrisWaterMaze.jpg}
\end{center}
  
\pause

Und was passiert derweil im Gehirn?
  
\end{frame}


 
%% %% TLIA



%% %% Learning Objectives
 
\begin{frame}

 \frametitle{Nach dieser Vorlesung sollten Sie folgendes können}



\begin{block}{Grundlagen:}
\begin{itemize}
\item
assoziative und nicht-assoziative Formen des Lernens beschreiben
\item
operante und klassische Konditionierung unterscheiden
\item
verschiedene Gedächtnisformen voneinander unterscheiden
\item
erklären, welche Gehirnstrukturen dem Gedächtnis zugrunde liegen
\item
Aufmerksamkeit definieren und die zugrundeliegenden Vorgänge im Gehirn beschreiben
\item
erklären, was eine Hebbsche Synapse ist
\item
die Eigenschaften von Langzeitpotenzierung (LTP) erklären
\item
die zellulären und molekularen Prozesse bei LTP erläutern
\end{itemize}

\end{block}



\end{frame}


 
\begin{frame}

 \frametitle{Nach dieser Vorlesung sollten Sie folgendes können}

\begin{block}{Klinik:}
\begin{itemize}
\item
Formen von Amnesie aufzählen und erläutern
\item
Neglect erklären und typische Symptome beschreiben
\item
häufig verwendete Tests für Lernen und Gedächtnis erläutern


\end{itemize}

\end{block}



\end{frame}










%% %% %% Main Body
 
 




 
 \section{Gedächtnis}

\begin{frame}{Was ist Gedächtnis?}

Gedächtnis ist die Einheit von Merkfähigkeit und Erinnerung. 

(Information muss gespeichert und wieder aufgerufen werden). 

\pause

Dies geschieht durch die Bildung von Engrammen ("Gedächtnisspuren") zur Speicherung von Wahrnehmungen und Erfahrungen.​

\end{frame}

%% Engramme müssen niciht perfekt sein

%% Emanuel Folien zu Gedächtnisformen


\section{Aufmerksamkeit}

%% Allgemeines zur Aufmerksamkeit

{ \usebackgroundtemplate{\includegraphics[width=1.2\paperwidth]{Aufmerksamkeit.pdf}} 
\begin{frame}



\end{frame} 
}

%% Steuerung von Aufmerksamkeit

%% ADHS

%% Einseitige Aufmerksamkeit



 \section{Lernen}
 
\section{Synaptische Plastizität}

% %% Fire together - wire together
% %% Hebb, Ramon y Cajal
\begin{frame}
\frametitle{Wie ist Gedächtnis auf zellulärer Ebene codiert?}
(* \dots die obligatorische  Ramon y Cajal Folie)
\begin{columns}[c]
\begin{column}{5cm}
\begin{itemize}
\item
Santiago Ram{\'o}n y Cajal (1852-1934)
\item
Das Nervensystem besteht aus Zellen, die miteinander kommunizieren
\item
Beim Lernen verändern sich die Verbindungen zwischen Nervenzellen 
\end{itemize}
\end{column}

\begin{column}{5cm}

\includegraphics[width=\textwidth]{cajal_embroidery_project.jpg}


\end{column}

\end{columns}

\end{frame}




\begin{frame}
\frametitle{Wie ist Gedächtnis auf zellulärer Ebene codiert?}


\begin{block}{Hebbsche Lernregel (nach Donald Hebb, 1904-1985):}
\begin{quote}
Wenn ein Axon der Zelle A [\dots] Zelle B erregt und wiederholt und dauerhaft zur Erzeugung von Aktionspotentialen in Zelle B beiträgt, so resultiert dies in Wachstumsprozessen oder metabolischen Veränderungen in einer oder in beiden Zellen, die bewirken, dass die Effizienz von Zelle A in Bezug auf die Erzeugung eines Aktionspotentials in B größer wird

\end{quote}

(D.O. Hebb. The Organisation of Behaviour. 1949.)
\end{block}

\pause


\begin{block}{Hebbsche Lernregel (Kurzfassung)}
%% Spoiler
\pause

Neurons that fire together wire together
\end{block}

\end{frame}


% %% Ephys stuff - Lomo, Bliss
\begin{frame}
\frametitle{Was ist Langzeitpotenziierung?}

Auch Long Term Potentiation (LTP) genannt, elektrophysiologisches Phänomen entdeckt von Terje L{\o}mo und Timothy Bliss (1973)

\begin{block}{Typisches Induktionsprotokoll für LTP}

\begin{itemize}
\item
Hippocampus
\item
Eine Synapse wird regelmäßig stimuliert (z.B. alle 30s), gemessen wird die Steigung des EPSP
\item
Nach einiger Zeit erfolgt eine kurze hochfrequente Stimulation (z.B. 100 Hz für 1s)
\item
Danach wird wieder alle 30s stimuliert und die Steigung des EPSP gemessen
\end{itemize}

\end{block}

\end{frame}



\begin{frame}
\frametitle{Was ist Langzeitpotenziierung?}

\begin{center}
\includegraphics<1>[width=0.8\textwidth]{LTP_exemplar_WMC_1.png}

\includegraphics<2>[width=0.8\textwidth]{LTP_exemplar_WMC_2.png}

\includegraphics<3>[width=0.8\textwidth]{LTP_exemplar_WMC_3.png}

\includegraphics<4>[width=0.8\textwidth]{LTP_exemplar_WMC_4.png}

\includegraphics<5>[width=0.8\textwidth]{LTP_exemplar_WMC_5.png}

\includegraphics<6>[width=0.8\textwidth]{LTP_exemplar_WMC_6.png}
\end{center}


\end{frame}


% %% Model for memory, not memory itself
\begin{frame}
\frametitle{Wie hängen LTP und Gedächtnis zusammen?}

\begin{block}{Überlegen Sie }

\begin{itemize}
\item
Was sagt uns die Existenz von LTP über Gedächtnis?
\item
Kann LTP Gedächtnis erklären?
\item
Welche weiteren Informationen würden Sie benötigen?
\end{itemize}

\end{block}
\end{frame}

% %% But Rats and stuff
% %% Spoiler
\begin{frame}
\frametitle{LTP als ein Modell für Gedächtnis}



\begin{columns}[c]

\begin{column}{7cm}
\begin{block}{Hinweise auf eine Verbindung zwischen LTP und Gedächtnis (Tierversuch)}

\begin{itemize}
\item
Pharmakologische Mittel die LTP blockieren, blockieren auch die Performance bei Aufgaben zum räumlichen Lernen
% (e.g. Morris 1986; Tonegawa 1996)
\item
Ein Knockout mancher Proteine führt zu Defiziten sowohl von LTP als auch beim Lernen
% (e.g. Silva et al. 1992)
\item
Lernen führt zu denselben molekularen Veränderungen wie LTP
% (Whitlock 2006)
\item
Bei Tieren, die bereits eine Lernphase durchlaufen haben kann (in den entsprechenden Neuronen) keine weitere LTP induziert werden.
% (Whitlock 2006)
\end{itemize}
\end{block}


\end{column}



\begin{column}{3cm}
\begin{center}
\includegraphics[width=\textwidth]{MorrisWaterMaze.jpg}
\end{center}

\textbf{Aber} das heißt noch nicht dass LTP für alle Gedächtnisformen notwendig oder hinreichend ist. 

\end{column}


\end{columns}

\end{frame}





\begin{frame}
\frametitle{Eigenchaften von LTP}

\begin{block}{LTP ist \dots}

\begin{itemize}
\item
Persistent
\item
Assoziativ/kooperativ
\item
Input-spezifisch
\end{itemize}
\end{block}


\end{frame}

% %% Persistence

\begin{frame}
\frametitle{Eigenschaften LTP}


\begin{block}{LTP ist persistent}

\textcolor{thene}{Wie lange "hält" LTP? Warum ist das wichtig?}



%% Spoiler
\pause


\begin{center}
\includegraphics[width=0.8\textwidth]{LTP_exemplar_WMC.jpg}
\end{center}








\end{block}


\end{frame}

% %% Associative/cooperative

\begin{frame}
\frametitle{Eigenschaften von LTP}

\begin{block}{LTP ist assoziative/kooperativ}

Schwache Stimulierung einer Synapse an sich kann keine LTP auslösen.

Aber LTP ist \dots

\pause

\begin{columns}[c]
\begin{column}{5cm}

\begin{itemize}
\item
\textbf{Assoziativ}: LTP kann auftreten, wenn gleichzeitig zur schwachen Stimulierung irgendwo in der Nähe eine starke Stimulierung stattfindet.
\item
\textbf{Kooperativ}: LTP kann auftreten, wenn mehrere benachbarte Synapsen schwach stimuliert werden. 
\end{itemize}

\end{column}

\begin{column}{5cm}

\begin{center}
\includegraphics[width=0.8\textwidth]{/home/melanie/Work/pictures/memory/LTP_associative_modified.png}
\end{center}

\end{column}

\end{columns}    

  
\end{block}


\end{frame}

% %% Input-specificity 

\begin{frame}
\frametitle{Eigenschaften von LTP}

\begin{block}{LTP ist Input-specifisch}

\begin{columns}[c]
\begin{column}{5cm}

\begin{center}
\includegraphics[width=0.8\textwidth]{/home/melanie/Work/pictures/memory/LTP_input_specific_modified.png}
\end{center}


\end{column}

\begin{column}{5cm}

\pause 
\begin{itemize}
\item
LTP passiert nur an der Synapse, wo sie induziert wurde.
\item
LTP wird \textbf{nicht} von einer Synapse zur nächsten übertragen. 
\end{itemize}

\end{column}

\end{columns}    


\end{block}


\end{frame}


% %% Stages
\begin{frame}
\frametitle{Stadien LTP}

\pause 

\begin{block}{Frühe Phase}
\begin{itemize}
\item
Aktivierung von NMDA Rezeptoren führt zum Influx von Kalzium 
\item
Kalzium aktiviert CaMKII, CaMKII kann dann stabil aktiv bleiben
\item
CaMKII erhöht die Anzahl und Aktivität von AMPA Rezeptoren
\item
CaMKII aktiviert weitere Signaltransduktionswege. 
\item
Das Aktin-Zytoskelett wird umgebaut und die Form und Größe des dendriticschen Dornenfortsatzes verändert sich
\end{itemize}
\end{block}

\pause

\begin{block}{Spätphase}
\begin{itemize}
\item
Weitere Proteine werden durch lokale Synthese rekrutiert
\item
Signale an Transkriptionsfaktoren führen zu Veränderungen der Genexpression
\item
ynaptic tagging and capture
\end{itemize}
\end{block}
 

\end{frame}



% %% %% Stages and Molecular basis of LTP
% \section{Early LTP}

% \subsection{NMDA receptors as coincidence detectors}


% %% AMPAR, Mg block, NMDAR, 

% \begin{frame}
% \frametitle{NMDAR activation}

% \begin{center}
% \includegraphics<1>[width=\textwidth]{/home/melanie/Work/pictures/LTP/LTP1_AMPAR.png}
% \includegraphics<2>[width=\textwidth]{/home/melanie/Work/pictures/LTP/LTP2_depol.png}
% \includegraphics<3>[width=\textwidth]{/home/melanie/Work/pictures/LTP/LTP3_NMDAR.png}
% \includegraphics<4>[width=\textwidth]{/home/melanie/Work/pictures/LTP/LTP4_Ca.png}
% \includegraphics<5>[width=\textwidth]{/home/melanie/Work/pictures/LTP/LTP5_Calm.png}
% \end{center}

% \end{frame}

% \begin{frame}
% \frametitle{NMDAR activation}

% \begin{itemize}
% \item
% Both AMPA receptors (AMPAR) and NMDA receptors (NMDAR) are activated by Glutamate, but NMDAR is usually blocked by Magnesium. 
% \item
% AMPAR activation leads to membrane depolarisation, which relieves the Mg\textsuperscript{2+} block
% \item
% Without the Mg block, a subsequent Glutamate signal can activate NMDAR and allow Ca influx into the postsynaptic neuron
% \end{itemize}

% \end{frame}


% \begin{frame}
% \frametitle{NMDA receptor as a coincidence detector}


% \begin{columns}[c]
% \begin{column}{5cm}
% \includegraphics[width=\textwidth]{/home/melanie/Work/pictures/LTP/NMDA_receptor.jpg}
% \end{column}

% \begin{column}{5cm}
% \begin{itemize}
% \item
% What do people mean when they say that NMDAR is a ``coincidence detector''?
% \item
% How is this related to Hebb's law?
% \item
% \textcolor{theme}{What property of LTP may this explain? }
% \end{itemize}
% \end{column}

% \end{columns}

% \end{frame}

% \subsection{Postsynaptic Calcium signalling}

% \begin{frame}
% \frametitle{Postsynaptic Calcium signalling}


% \begin{center}
% \includegraphics[width=0.8\textwidth]{/home/melanie/Work/pictures/signalling/signalling_general_SBGN_compliant.png}
% \end{center}

% \end{frame}






% \begin{frame}
%   \frametitle{Postsynaptic Calcium signalling}
%   \begin{itemize}
% \item
% Calcium activates calmodulin
% \end{itemize}


% \begin{block}{At high Calcium levels \dots}
% \begin{itemize}
% \item
% Calmodulin preferentially activates CaMKII
% \item
% CaMKII phosphorylates AMPA receptors, which has two effects:
% \begin{itemize}
% \item
% Increases AMPAR conductance
% \item
% Increases number of AMPARs in the PSD
% \end{itemize}
% \item
% CaMKII also phosphorylates and thereby activates itself
% \item
% Long-term potentiation (LTP) of the synapse
% \end{itemize}
% \end{block}



% \begin{block}{At moderate Calcium levels \dots}
% \begin{itemize}
% \item
% Calmodulin preferentially activates protein phosphatase 2B (PP2B)
% \item
% PP2B activation leads to dephosphorylation of CaMKII and AMPARs
% \item
% Net AMPAR phosphorylation decreases, leading to long-term depression (LTD) of the synapse
% \end{itemize}
% \end{block}


% \end{frame}

% \begin{frame}
% \frametitle{Bistable system (``on'' or ``off'')}

% \begin{columns}[c]


% \begin{column}{5cm}

% AMPA receptor activation

% $\;$\\[1cm]

% \includegraphics[width=\textwidth]{/home/melanie/Work/pictures/multi-scale/AMPAR-Ca-Scan.png}
% \end{column}

% \begin{column}{5cm}

% Bistability of kinases and phosphatases

% $\;$\\[1cm]


% \includegraphics[width=\textwidth]{/home/melanie/Work/pictures/multi-scale/Key-Species-Scan.png}
% \end{column}

% \end{columns}

% $\;$ \\[0.5 cm](Work by Yubin Xie)

% \end{frame}
 


% %% %% CaMKII
% \begin{frame}
% \frametitle{A closer look at CaMKII activation}
% \begin{center}
% \includegraphics[width=0.8\textwidth]{/home/melanie/Work/pictures/signalling/signalling_general_SBGN_compliant.png}
% \end{center}

% \end{frame}
 

% \begin{frame}
% \frametitle{A closer look at CaMKII activation}

% \centering
% \includegraphics<1>[height=0.9\textheight]{/home/melanie/Work/pictures/metaphore/Victorinox_open.jpg}

% \includegraphics<2>[width=4cm]{/home/melanie/Work/pictures/CaMKII/CaMKII_monomer_gray.png}

% \end{frame}

% \begin{frame}
% \frametitle{A closer look at CaMKII activation}

 
% \begin{center}
% \includegraphics[width=0.5\textwidth]{/home/melanie/Work/pictures/CaMKII/dodec_grayred.png}
% \end{center} 

% \end{frame}



% \begin{frame}
% \frametitle{A closer look at CaMKII activation}

% \begin{itemize}
% \item
% CaMKII activates AMPA receptors, but also a range of other synaptic proteins
% \item
% CaMKII is active if it is open (inhibitory helix is away from its active site)
% \item
% Calmodulin binding stabilises the open state
% \item
% Phosphorylation also stabilises the open state
% \item
% But CaMKII is a kinase, so it can phosphorylate itself!
% \item
% More precisely, a CaMKII subunit phosphorylates its neighbour in a ring of six
% \item
% This makes CaMKII activity Calcium-independent
% \item
% \textcolor{theme}{What property of LTP is this important for?}
% \end{itemize}
% \end{frame}



% \subsection{Changes in spine shape}


% %% actin remodelling
% \begin{frame}
% \frametitle{Changes in spine shape}

% Molecular changes in early LTP also lead to remodelling of the actin cytoskeleton and changes in the size and shape of the dendritic spine. 
 
% \includegraphics[width=\textwidth]{/home/melanie/Work/pictures/LTP/ActinRemodelingFigure.jpg}

% \pause

% This stabilises potentiated spines and contributes to LTP persistence
  
% \end{frame}

% \begin{frame}
% \frametitle{Role of Calcium signalling in actin remodelling}

% \begin{center}
% \includegraphics<1>[width=0.6\textwidth]{/home/melanie/Work/pictures/LTP/spine_plasticity_low_Ca.png}
% \includegraphics<2>[width=0.6\textwidth]{/home/melanie/Work/pictures/LTP/spine_plasticity_high_Ca.png}
% \end{center}

% \end{frame}


% \section{Later stages of LTP}


% %% local protein translation
% \begin{frame}
% \frametitle{Late LTP}

% \begin{itemize}
% \item
% LTP is maintained over time scales longer than protein turnover. \textcolor{theme}{How?}
% \pause
% %% Spoiler
% \item
% LTP relies on additional protein translation
% \item
% Local protein synthesis ensures specificity
% \item
% At longer timescales, LTP maintenance probably involves changes in gene expression.
% \item
% This is achieved through signalling pathways (e.g. ERK) that signal to transcription factors.
% \item
% \textcolor{theme}{\emph{Why is this difficult?}}
% \end{itemize}

% \end{frame}

% %% Spoiler
% \begin{frame}
% \frametitle{(Even later) late LTP}

% \begin{center}
% \includegraphics[width=0.6\textwidth]{/home/melanie/Work/pictures/brain/Complete_neuron_cell_diagram_en.png}
% \end{center}


% \pause

% How does the newly created mRNA/protein know where to go?  

% \pause
% \textcolor{theme}{\(\rightarrow \) Tutorial! }

% \end{frame}


 


%% %% %% %% Review

\begin{frame}

 \frametitle{Jetzt* sollten Sie folgendes können}



\begin{block}{Grundlagen:}
\begin{itemize}
\item
assoziative und nicht-assoziative Formen des Lernens beschreiben
\item
operante und klassische Konditionierung unterscheiden
\item
verschiedene Gedächtnisformen voneinander unterscheiden
\item
erklären, welche Gehirnstrukturen dem Gedächtnis zugrunde liegen
\item
Aufmerksamkeit definieren und die zugrundeliegenden Vorgänge im Gehirn beschreiben
\item
erklären, was eine Hebbsche Synapse ist
\item
die Eigenschaften von Langzeitpotenzierung (LTP) erklären
\item
die zellulären und molekularen Prozesse bei LTP erläutern
\end{itemize}

\end{block}



\end{frame}


 
\begin{frame}

 \frametitle{Jetzt* sollten Sie folgendes können}

\begin{block}{Klinik:}
\begin{itemize}
\item
Formen von Amnesie aufzählen und erläutern
\item
Neglect erklären und typische Symptome beschreiben
\item
häufig verwendete Tests für Lernen und Gedächtnis erläutern


\end{itemize}

\end{block}



\end{frame}




%% %% %% %% Feedbackhinweisblock

\begin{frame}
\frametitle{Danke für Ihr Feedback!}

\begin{columns}[c]

\begin{column}{6cm}
\begin{center}
 \includegraphics[width=\textwidth]{smilie_balloons.jpg}
\end{center}

\end{column}

\begin{column}{4cm}


\begin{center}
\includegraphics[width=\textwidth]{feedback_QR.png}
\end{center}
\end{column}


\end{columns}

\end{frame}



%% %% %% Bildnachweis
\begin{frame}
\frametitle{Bildnachweis}
\begin{tiny}

Teile dieser Vorlesung wurden übernommen von einer Vorlesung von Prof. Emanuel Busch,  Health and Medical University Potsdam, dem wir an dieser Stelle herzlich danken. Wo nicht anders gekennzeichnet, stammen Abbildungen aus dieser Vorlesung.  


 
\begin{itemize}

\item
Gestickte Versionen von Neuronen-Zeichnungen von Santiago Ramon y Cajal. From: A.R. Mehta, C.M. Abbott, S. Chandran, and J.E. Haley (2020). The Cajal Embroidery Project: celebrating neuroscience. \emph{The Lancet Neurology}, 19(12), p.979.


%% all lectures
\item
Luftballons mit frohen und traurigen Smilies. Photo by \href{https://unsplash.com/@artbyhybrid?utm_source=unsplash&utm_medium=referral&utm_content=creditCopyText}{Hybrid} on \href{https://unsplash.com/s/photos/feedback?utm_source=unsplash&utm_medium=referral&utm_content=creditCopyText}{Unsplash}
%%%%%%%%%%%

\item
Ratte im Morris Water Maze. Jean-Etienne Minh-Duy Poirrier, CC BY-SA 2.0 \url{https://creativecommons.org/licenses/by-sa/2.0}, via Wikimedia Commons

\end{itemize}
\end{tiny}
\end{frame}






\end{document}

%%% Frequently used snippets

%% \begin{columns}[c]

%% \begin{column}{5cm}
%% \end{column}

%% \begin{column}{5cm}
%% \end{column}


%% \end{columns}




