\documentclass{beamer}	
\mode<presentation>
 
\usepackage{pdfpages}
\usepackage{fancyvrb}
\usepackage{chemarr}

\usepackage{amsmath}		%% mathematics typesetting
\usepackage{amssymb}
 
\usepackage{epigraph}   %% nice setting of quotations

\usepackage{tabularx} %% allows to use row colours in tables

\usepackage{ulem}

\usepackage{booktabs}

\usepackage{siunitx} %% tpyeset SI units

\usepackage{CJKutf8} %% typeset Chinese characters

\usepackage{pdfpages}%% include pdfs

\usepackage{graphicx}
\usepackage{animate} %% show animated gifs

\DeclareMathAlphabet{\mathcalligra}{T1}{calligra}{m}{n}


% Color and Theme. Can be changed. However, this one's quite nice.
\usetheme{Madrid}
\definecolor{theme}{rgb}{0.84,0,0.21}
\usecolortheme[named=theme]{structure}

%%  Title information
\title[M11.11.5 Lernen, Gedächtnis]{M11.11.5 Zentrales Nervensystem 3: \\ Lernen und Gedächtnis}
\author[melanie.stefan@medicalschool-berlin.de]{}
\institute[]{Prof. Melanie Stefan \\ melanie.stefan@medcialschool-berlin.de}
\date{SoSe 2022}
 

% Table of contents to pop up at the beginning of each section
\AtBeginSection[]
{
  \begin{frame}<beamer>
    \frametitle{Outline}
    \tableofcontents[currentsection,currentsubsection]
  \end{frame}
}
 
\beamertemplatenavigationsymbolsempty

\begin{document}


{ \usebackgroundtemplate{\includegraphics[width=1\paperwidth]{MSB_Titelseite.pdf}} 
\begin{frame}

 \maketitle 

$\,$\\[6cm] 


\end{frame} 
}


%% Hook + TLIA

{ \usebackgroundtemplate{\includegraphics[width=1.2\paperwidth]{miklos-kornyei-pWzCAm4jgzA-unsplash.jpg}} 
\begin{frame}
  Stimmt es eigentlich, dass man Fahrradfahren nicht verlernt?


$\;$\\[5.5cm]
\pause

Und was passiert \\ dabei im Gehirn?
  
\end{frame}
}

 
%% %% TLIA



%% %% Learning Objectives
 
\begin{frame}

 \frametitle{Nach dieser Vorlesung sollten Sie folgendes können}



\begin{block}{Grundlagen:}
\begin{itemize}
\item
assoziative und nicht-assoziative Formen des Lernens beschreiben
\item
operante und klassische Konditionierung unterscheiden
\item
verschiedene Gedächtnisformen voneinander unterscheiden
\item
erklären, welche Gehirnstrukturen dem Gedächtnis zugrunde liegen
\item
Aufmerksamkeit definieren und die zugrundeliegenden Vorgänge im Gehirn beschreiben
\item
erklären, was eine Hebbsche Synapse ist
\item
die Eigenschaften von Langzeitpotenzierung (LTP) erklären
\item
die zellulären und molekularen Prozesse bei LTP erläutern
\end{itemize}

\end{block}



\end{frame}


 
\begin{frame}

 \frametitle{Nach dieser Vorlesung sollten Sie folgendes können}

\begin{block}{Klinik:}
\begin{itemize}
\item
Formen von Amnesie aufzählen und erläutern
\item
Neglect erklären und typische Symptome beschreiben
\item
häufig verwendete Tests für Lernen und Gedächtnis erläutern


\end{itemize}

\end{block}



\end{frame}










%% %% %% Main Body
 
 




 
 \section{Gedächtnis}

\begin{frame}{Was ist Gedächtnis?}

Gedächtnis ist die Einheit von Merkfähigkeit und Erinnerung. 

(Information muss gespeichert und wieder aufgerufen werden). 

\pause

Dies geschieht durch die Bildung von Engrammen ("Gedächtnisspuren") zur Speicherung von Wahrnehmungen und Erfahrungen.​


\end{frame}

%% Engramme müssen niciht perfekt sein
\begin{frame}{Engramme müssen nicht perfekt sein!}

Sie brauchen keine genaue Fotografie vom Leipziger Platz im Gehirn \dots

\begin{center}
\includegraphics[width=0.8\paperwidth]{Leipziger_Platz_Berlin_01.jpg}   
\end{center}


\pause

\dots Sie müssen nur die MSB finden!
    
\end{frame}



%% Emanuel Folien zu Gedächtnisformen
%% 13-14
{ \usebackgroundtemplate{\includegraphics[width=\paperwidth]{Gedaechtnis_stadien.pdf}} 
\begin{frame}


 
\end{frame} 
}

{ \usebackgroundtemplate{\includegraphics[width=\paperwidth]{Ultrakurzzeitgedächtnis.pdf}} 
\begin{frame}



\end{frame} 
}


%% Kurzzeit Definition

\begin{frame}{Kurzzeitgedächtnis (auch: Arbeitsgedächtnis)}

Kurzer (Sekunden bis Minuten) "Arbeitsspeicher mit wenig Kapazität" 

\pause

Wie wenig Kapazität genau? \textcolor{theme}{Probieren wir es aus!}

\end{frame}


\begin{frame}{Kurzzeitgedächtnis (auch: Arbeitsgedächtnis)}


\huge{3 \quad 0 \quad 5} \\



\end{frame}


% \begin{frame}{Kurzzeitgedächtnis (auch: Arbeitsgedächtnis)}


% \huge{ \quad  \quad  \quad   \quad   } \\



% \end{frame}


% \begin{frame}{Kurzzeitgedächtnis (auch: Arbeitsgedächtnis)}


% \huge{ \quad  \quad  \quad   \quad  \quad \quad \quad  } \\



% \end{frame}




%% Emanuel Folie 15
{ \usebackgroundtemplate{\includegraphics[width=\paperwidth]{PFC.pdf}} 
\begin{frame}

\end{frame}
}


%% Langzeit: Arten + Kleiner Test


%% Langzeit: Hippocampus Emanuel 35-37, 

{ \usebackgroundtemplate{\includegraphics[width=\paperwidth]{Hippocampus.pdf}} 
\begin{frame}

\end{frame}
}


{ \usebackgroundtemplate{\includegraphics[width=\paperwidth]{hippocampus_verschaltung.pdf}} 
\begin{frame}

\end{frame}
}


{ \usebackgroundtemplate{\includegraphics[width=\paperwidth]{hippo_to_cortex.pdf}} 
\begin{frame}

\end{frame}
}
 

%% Beispiel: räumliches Lernen: Berlin Jonas Tebbe



%% Emanuel 39, 40

{ \usebackgroundtemplate{\includegraphics[width=\paperwidth]{watermaze.pdf}} 
\begin{frame}

\end{frame}
}

{ \usebackgroundtemplate{\includegraphics[width=\paperwidth]{place_grid_cells.pdf}} 
\begin{frame}

\end{frame}
}



\section{Aufmerksamkeit}

%% Allgemeines zur Aufmerksamkeit

{ \usebackgroundtemplate{\includegraphics[width=\paperwidth]{Aufmerksamkeit.pdf}} 
\begin{frame}



\end{frame} 
}


\begin{frame}{Aufmerksamkeit kann aus zwei Richtungen kommen:
}

\begin{itemize}
    \item
    \textbf{Top-down}: Bewusste Lenkung der Aufmerksamkeit auf einen Reiz, Vorgang oder Inhalt \\
    \emph{Beispiel: "Ich sollte in der Vorlesung aufpassen"}
\pause
    \item
    \textbf{Bottom-up}: Lenkung der Aufmerksamkeit auf einen Reiz durch den Reiz selber \\
    \emph{Beispiel: \pause Der eigene Name wird genannt}

\end{itemize}


    
\end{frame}


\begin{frame}{Multi-tasking geht schwer}

Aufmerksamkeit hat eine geringe Kapazität (vor allem, wenn es um ähnliche Prozesse geht). \pause

Beispiel: Benennen Sie die Farbe, in der die folgenden Wörter gedruckt sind:

\begin{tabular}{l l l}
\textcolor{blue}{schön}     & \textcolor{red}{neu}  & \textcolor{yellow}{klein} \\ \pause
\textcolor{gray}{rot}     & \textcolor{yellow}{grün}  & \textcolor{red}{blau} \\ 
\end{tabular}


\end{frame}


%% Steuerung von Aufmerksamkeit



%% ADHS

%% Einseitige Aufmerksamkeit



\section{Lernen}
%% Emanuel Folien 23,24,25

{ \usebackgroundtemplate{\includegraphics[width=\paperwidth]{arten_des_lernens.pdf}} 
\begin{frame}

\end{frame} 
}

{ \usebackgroundtemplate{\includegraphics[width=\paperwidth]{Habituation.pdf}} 
\begin{frame}

\end{frame} 
}

{ \usebackgroundtemplate{\includegraphics[width=\paperwidth]{Sensitivierung.pdf}} 
\begin{frame}

\end{frame} 
}

%% Klassische Konditionierung

\begin{frame}{Klassische Konditionierung}

\begin{columns}[c]

\begin{column}{5cm}
Ein neutraler Stimulus (der alleine keine Reaktion hervorrufen würde) wird wiederholt mit einem biologisch relevanten Stimulus (der eine Reaktion hervorruft) gepaart. Nach einiger Zeit ruft schon der neutrale Stimulus alleine die Reaktion hervor. (Klassisches Experiment: Pavlovsche Hunde)
\end{column}

\pause

\begin{column}{5cm}
\begin{center}
    \includegraphics[width=\textwidth]{Classical_Conditioning_Diagram.png}
\end{center}
\end{column}


\end{columns}

    
\end{frame}

%% Operante Konditionierung: Emanuel S. 27
{ \usebackgroundtemplate{\includegraphics[width=\paperwidth]{operand_conditioning.pdf} }
\begin{frame}

\end{frame} 
}


%% Modelllernen



%% Sensitive Phasen Emanuel S: 29
{ \usebackgroundtemplate{\includegraphics[width=\paperwidth]{sensitive_phasen.pdf} }
\begin{frame}

\end{frame} 
}



 
\section{Synaptische Plastizität}

% %% Fire together - wire together
% %% Hebb, Ramon y Cajal
\begin{frame}
\frametitle{Wie ist Gedächtnis auf zellulärer Ebene codiert?}
(* \dots die obligatorische  Ramon y Cajal Folie)
\begin{columns}[c]
\begin{column}{5cm}
\begin{itemize}
\item
Santiago Ram{\'o}n y Cajal (1852-1934)
\item
Das Nervensystem besteht aus Zellen, die miteinander kommunizieren
\item
Beim Lernen verändern sich die Verbindungen zwischen Nervenzellen 
\end{itemize}
\end{column}

\begin{column}{5cm}

\includegraphics[width=\textwidth]{cajal_embroidery_project.jpg}


\end{column}

\end{columns}

\end{frame}




\begin{frame}
\frametitle{Wie ist Gedächtnis auf zellulärer Ebene codiert?}


\begin{block}{Hebbsche Lernregel (nach Donald Hebb, 1904-1985):}
\begin{quote}
Wenn ein Axon der Zelle A [\dots] Zelle B erregt und wiederholt und dauerhaft zur Erzeugung von Aktionspotentialen in Zelle B beiträgt, so resultiert dies in Wachstumsprozessen oder metabolischen Veränderungen in einer oder in beiden Zellen, die bewirken, dass die Effizienz von Zelle A in Bezug auf die Erzeugung eines Aktionspotentials in B größer wird.

\end{quote}

(D.O. Hebb. The Organisation of Behaviour. 1949.)
\end{block}

\pause


\begin{block}{Hebbsche Lernregel (Kurzfassung)}
%% Spoiler
\pause

Neurons that fire together wire together
\end{block}

\end{frame}


% %% Ephys stuff - Lomo, Bliss
\begin{frame}
\frametitle{Was ist Langzeitpotenzierung?}

Auch Long Term Potentiation (LTP) genannt, elektrophysiologisches Phänomen, entdeckt von Terje L{\o}mo und Timothy Bliss (1973)

\begin{block}{Typisches Induktionsprotokoll für LTP}

\begin{itemize}
\item
Hippocampus
\item
Eine Synapse wird regelmäßig stimuliert (z.B. alle 30s), gemessen wird die Steigung des EPSP.
\item
Nach einiger Zeit erfolgt eine kurze hochfrequente Stimulation (z.B. 100 Hz für 1s).
\item
Danach wird wieder alle 30s stimuliert und die Steigung des EPSP gemessen.
\end{itemize}

\end{block}

\end{frame}



\begin{frame}
\frametitle{Was ist Langzeitpotenzierung?}

\begin{center}
\includegraphics<1>[width=0.8\textwidth]{LTP_exemplar_WMC_1.png}

\includegraphics<2>[width=0.8\textwidth]{LTP_exemplar_WMC_2.png}

\includegraphics<3>[width=0.8\textwidth]{LTP_exemplar_WMC_3.png}

\includegraphics<4>[width=0.8\textwidth]{LTP_exemplar_WMC_4.png}

\includegraphics<5>[width=0.8\textwidth]{LTP_exemplar_WMC_5.png}

\includegraphics<6>[width=0.8\textwidth]{LTP_exemplar_WMC_6.png}
\end{center}


\end{frame}


% %% Model for memory, not memory itself
\begin{frame}
\frametitle{Wie hängen LTP und Gedächtnis zusammen?}

\begin{block}{Überlegen Sie}

\begin{itemize}
\item
Was sagt uns die Existenz von LTP über Gedächtnis?
\item
Kann LTP Gedächtnis erklären?
\item
Welche weiteren Informationen würden Sie benötigen?
\end{itemize}

\end{block}
\end{frame}

% %% But Rats and stuff
% %% Spoiler
\begin{frame}
\frametitle{LTP als ein Modell für Gedächtnis}



\begin{columns}[c]

\begin{column}{7cm}
\begin{block}{Hinweise auf eine Verbindung zwischen LTP und Gedächtnis (Tierversuch)}

\begin{itemize}
\item
Pharmakologische Mittel, die LTP blockieren, blockieren auch die Performance bei Aufgaben zum räumlichen Lernen
% (e.g. Morris 1986; Tonegawa 1996)
\item
Ein Knockout mancher Proteine führt zu Defiziten sowohl von LTP als auch beim Lernen.
% (e.g. Silva et al. 1992)
\item
Lernen führt zu denselben molekularen Veränderungen wie LTP.
% (Whitlock 2006)
\item
Bei Tieren, die bereits eine Lernphase durchlaufen haben, kann (in den entsprechenden Neuronen) keine weitere LTP induziert werden.
% (Whitlock 2006)
\end{itemize}
\end{block}


\end{column}



\begin{column}{3cm}
\begin{center}
\includegraphics[width=\textwidth]{MorrisWaterMaze.jpg}
\end{center}

\textbf{Aber} das heißt noch nicht, dass LTP für alle Gedächtnisformen notwendig oder hinreichend ist. 

\end{column}


\end{columns}

\end{frame}





\begin{frame}
\frametitle{Eigenchaften von LTP}

\begin{block}{LTP ist \dots}

\begin{itemize}
\item
Persistent
\item
Assoziativ/kooperativ
\item
Input-spezifisch
\end{itemize}
\end{block}


\end{frame}

% %% Persistence

\begin{frame}
\frametitle{Eigenschaften LTP}


\begin{block}{LTP ist persistent}

\textcolor{thene}{Wie lange "hält" LTP? Warum ist das wichtig?}



%% Spoiler
\pause


\begin{center}
\includegraphics[width=0.8\textwidth]{LTP_exemplar_WMC.jpg}
\end{center}








\end{block}


\end{frame}

% %% Associative/cooperative

\begin{frame}
\frametitle{Eigenschaften von LTP}

\begin{block}{LTP ist assoziative/kooperativ}

Schwache Stimulierung einer Synapse an sich kann keine LTP auslösen.

Aber LTP ist \dots

\pause

\begin{columns}[c]
\begin{column}{5cm}

\begin{itemize}
\item
\textbf{Assoziativ}: LTP kann auftreten, wenn gleichzeitig zur schwachen Stimulierung irgendwo in der Nähe eine starke Stimulierung stattfindet.
\item
\textbf{Kooperativ}: LTP kann auftreten, wenn mehrere benachbarte Synapsen schwach stimuliert werden. 
\end{itemize}

\end{column}

\begin{column}{5cm}

\begin{center}
\includegraphics[width=0.8\textwidth]{LTP_associative_modified.png}
\end{center}

\end{column}

\end{columns}    

  
\end{block}


\end{frame}

% %% Input-specificity 

\begin{frame}
\frametitle{Eigenschaften von LTP}

\begin{block}{LTP ist Input-spezifisch}

\begin{columns}[c]
\begin{column}{5cm}

\begin{center}
\includegraphics[width=0.8\textwidth]{LTP_input_specific_modified.png}
\end{center}


\end{column}

\begin{column}{5cm}

\pause 
\begin{itemize}
\item
LTP passiert nur an der Synapse, wo sie induziert wurde.
\item
LTP wird \textbf{nicht} von einer Synapse zur nächsten übertragen. 
\end{itemize}

\end{column}

\end{columns}    


\end{block}


\end{frame}


% %% Stages
\begin{frame}
\frametitle{Stadien LTP}

\pause 

\begin{block}{Frühe Phase}
\begin{itemize}
\item
Aktivierung von NMDA Rezeptoren führt zum Influx von Kalzium 
\item
Kalzium aktiviert CaMKII, CaMKII kann dann stabil aktiv bleiben
\item
CaMKII erhöht die Anzahl und Aktivität von AMPA Rezeptoren
\item
CaMKII aktiviert weitere Signaltransduktionswege. 
\item
Das Aktin-Zytoskelett wird umgebaut und die Form und Größe des dendritischen Dornenfortsatzes verändert sich
\end{itemize}
\end{block}

\pause

\begin{block}{Spätphase}
\begin{itemize}
\item
Weitere Proteine werden durch lokale Synthese rekrutiert
\item
Signale an Transkriptionsfaktoren führen zu Veränderungen der Genexpression
\item
Synaptic tagging and capture
\end{itemize}
\end{block}
 

\end{frame}



% %% %% Stages and Molecular basis of LTP


%% AMPAR, Mg block, NMDAR, 

\begin{frame}
\frametitle{Aktivierung von NMDA Rezeptoren}

\begin{center}
\includegraphics<1>[width=\textwidth]{LTP1_AMPAR.png}
\includegraphics<2>[width=\textwidth]{LTP2_depol.png}
\includegraphics<3>[width=\textwidth]{LTP3_NMDAR.png}
\includegraphics<4>[width=\textwidth]{LTP4_Ca.png}
\includegraphics<5>[width=\textwidth]{LTP5_Calm.png}
\end{center}

\end{frame}

\begin{frame}
\frametitle{Aktivierung von NMDA Rezeptoren}

\begin{itemize}
\item
Sowohl AMPA Rezeptoren als auch NMDA Rezeptoren werden duch Glutamate aktiviert, but NMDA Rezeptoren sind meistens (bei Ruhe-Membranpotenzial) von Mg\textsuperscript{2+} blockiert. 
\item
Aktivierung von AMPAR führt zu Depolarisierung der Membran, wodurch die Mg\textsuperscript{2+} Blockade aufgehoben wird.
\item
Ohne diese Magnesium-Blockade kann ein neues Glutamat-Signal NMDA Rezeptoren aktivieren, es kommt zum Einfluss von Ca\textsuperscript{2+} in das postsynaptische Neuron
\end{itemize}

\end{frame}


\begin{frame}

\frametitle{NMDA Rezeptoren als "Coincidence Detectors"} 


\begin{columns}[c]
\begin{column}{5cm}
\includegraphics[width=\textwidth]{NMDA_receptor.jpg}
\end{column}

\begin{column}{5cm}
\begin{itemize}
\item
Was ist gemeint, wenn der NMDA Rezeptor als "Coincience detector" ("Gleichzeitigkeit-Detektor") bezeichnet wird?
\item
Wie hängt das mit der Hebbschen Lernregel zusammen?
\item
Welche Eigenschaft von LTP wird dadurch erklärt?
\end{itemize}
\end{column}

\end{columns}

\end{frame}


\begin{frame}
\frametitle{Kalzium-Signaltransduktion im postsynaptischen Neuron}


\begin{center}
\includegraphics[width=0.8\textwidth]{/home/melanie/Work/pictures/signalling/signalling_general_SBGN_compliant.png}
\end{center}

\end{frame}



\begin{frame}

\frametitle{Kalzium-Signaltransduktion im postsynaptischen Neuron}
  \begin{itemize}
\item
Ca\textsuperscript{2+} aktiviert Calmodulin
\end{itemize}


\begin{block}{Bei hohen Ca\textsuperscript{2+} Konzentrationen \dots}
\begin{itemize}
\item
Calmodulin aktiviert (eher) CaMKII (Ca\textsuperscript{2+}/Calmodulin-dependent kinase II)
\item
CaMKII phosphoryliert AMPA Rezeptoren, was zwei Auswirkungen hat:
\begin{itemize}
\item
Leitfähigkeit von AMPA Rezeptoren nimmt zu
\item
Anzahl von AMPA Rezeptoren in der postsynaptischen Zellmembran nimmt zu
\end{itemize}
\item
CaMKII phosphoryliert sich auch selber und aktiviert sich dadurch
\item
Die Synapse wird langfristig verstärkt (LTP)
\end{itemize}
\end{block}



\begin{block}{Bei weniger hohen Ca\textsuperscript{2+} Konzentrationen \dots}
\begin{itemize}
\item
Calmodulin aktiviert (eher) Protein Phosphatase 2B (PP2B)
\item
Aktivierung von PP2B führt zu Dephosphorylierung von AMPA Rezeptoren und von CaMKII 
\item
Insgesamt verringert sich die Phosphorylierung von AMPA Rezeptoren, und die Synapse wird langfristig geschwächt (Long Term Depression, LTD)
\end{itemize}
\end{block}


\end{frame}

\begin{frame}
\frametitle{Bistabiles System ("Ein"-"Aus"-Schalter)}

\begin{columns}[c]


\begin{column}{5cm}

AMPA Rezeptor Aktivierung 

$\;$\\[1cm]

\includegraphics[width=\textwidth]{/home/melanie/Work/pictures/multi-scale/AMPAR-Ca-Scan.png}
\end{column}

\begin{column}{5cm}

Bistabilität von Kinasen und Phosphatasen

$\;$\\[1cm]


\includegraphics[width=\textwidth]{/home/melanie/Work/pictures/multi-scale/Key-Species-Scan.png}
\end{column}

\end{columns}

$\;$ \\[0.5 cm](Arbeit von Yubin Xie)

\end{frame}
 


% %% %% CaMKII
\begin{frame}
\frametitle{CaMKII Aktivierung: Detail}
\begin{center}
\includegraphics[width=0.8\textwidth]{signalling_general_SBGN_compliant.png}
\end{center}

\end{frame}
 

\begin{frame}
\frametitle{CaMKII Aktivierung: Detail}

\centering
\includegraphics<1>[height=0.9\textheight]{Victorinox_open.jpg}

\includegraphics<2>[width=4cm]{CaMKII_monomer_gray.png}

\end{frame}

\begin{frame}
\frametitle{CaMKII Aktivierung: Detail}

 
\begin{center}
\includegraphics[width=0.5\textwidth]{dodec_grayred.png}
\end{center} 

\end{frame}



\begin{frame}
\frametitle{CaMKII Aktivierung: Detail}

\begin{itemize}
\item
CaMKII aktiviert AMPA Rezeptoren, aber auch andere synaptische Proteine
\item
CaMKII ist aktiv wenn sie in der "offenen" Konformation vorliegt (die inhibitorische Helix ist nicht ans aktive Zentrum gebunden)
\item
Bindung von Calmodulin stabilisiert die aktive Konformation
\item
Phosphorylierung stabilisiert ebenfalls die aktive Konformation
\item
Aber CaMKII ist eine Kinase! D.h. sie kann sich selber phosphorylieren! (Genauer: Eine Untereinheit im 6-er Ring phosphoryliert eine benachbarte Untereinheit) 
\item
Durch diese Phosphorylierung wird die Aktivität von CaMKII unabhängig von Kalzium
\item
\textcolor{theme}{Für welche Eigenschaft der LTP ist das wichtig?}
\end{itemize}
\end{frame}


% %% actin remodelling
\begin{frame}
\frametitle{Veränderungen in der Struktur des dendritischen Dornfortsatzese}

Molekulare Veränderungen in der frühen Phase von LTP führen dazu, dass das Aktin-Zytoskelett sich umbaut und der dendritische Dornfortsatz wächst.

\includegraphics[width=\textwidth]{ActinRemodelingFigure.jpg}

\pause
Dadurch werden potenzierte Dornfortsätze stabilisiert, was zur Persistenz der LTP beiträgt


\end{frame}

\begin{frame}
\frametitle{Auch das passiert durch Kalzium-Signaltransduktion}

\begin{center}
\includegraphics<1>[width=0.6\textwidth]{/home/melanie/Work/pictures/LTP/spine_plasticity_low_Ca.png}
\includegraphics<2>[width=0.6\textwidth]{/home/melanie/Work/pictures/LTP/spine_plasticity_high_Ca.png}
\end{center}

\end{frame}




% %% local protein translation
\begin{frame}
\frametitle{Spätphase der LTP}

\begin{itemize}
\item
LTP kann über Zeiträume aufrecht erhalten werden, die länger sind als die Lebenszeit einzelner Proteine. \textcolor{theme}{Wie?}
\pause
%% Spoiler
\item
LTP braucht zusätzliche Produktion von Proteinen
\item
Lokale Proteinsynthese (aus zirkulierender mRNA) gewährleistet Spezifität
\item
Längerfristig sind aber wahrscheinlich Veränderungen in der Genexpression notwendig
\item
Daher werden durch LTP auch Signalkaskaden (z.B. ERK) aktiviert, die in den Zellkern wirken 
\item
\textcolor{theme}{\emph{Warum ist das schwer?}}
\end{itemize}

\end{frame}

% %% Spoiler
\begin{frame}
\frametitle{(Späte) Spätphase der LTP}

\begin{center}
\includegraphics[width=0.6\textwidth]{Complete_neuron_cell_diagram_en.png}
\end{center}


\pause

Woher weiß neu hergestellte mRNA (oder neu hergestelltes Protein), wo sie hin müssen?   \\ 
\pause "Synaptic tagging and capture" - Molekulare Markierung, die zirkulierender mRNA anzeigt, dass hier eine Synapse ist.



\end{frame}


 


%% %% %% %% Review

\begin{frame}

 \frametitle{Jetzt* sollten Sie folgendes können}



\begin{block}{Grundlagen:}
\begin{itemize}
\item
assoziative und nicht-assoziative Formen des Lernens beschreiben
\item
operante und klassische Konditionierung unterscheiden
\item
verschiedene Gedächtnisformen voneinander unterscheiden
\item
erklären, welche Gehirnstrukturen dem Gedächtnis zugrunde liegen
\item
Aufmerksamkeit definieren und die zugrundeliegenden Vorgänge im Gehirn beschreiben
\item
erklären, was eine Hebbsche Synapse ist
\item
die Eigenschaften von Langzeitpotenzierung (LTP) erklären
\item
die zellulären und molekularen Prozesse bei LTP erläutern
\end{itemize}

\end{block}



\end{frame}


 
\begin{frame}

 \frametitle{Jetzt* sollten Sie folgendes können}

\begin{block}{Klinik:}
\begin{itemize}
\item
Formen von Amnesie aufzählen und erläutern
\item
Neglect erklären und typische Symptome beschreiben
\item
häufig verwendete Tests für Lernen und Gedächtnis erläutern


\end{itemize}

\end{block}



\end{frame}




%% %% %% %% Feedbackhinweisblock

\begin{frame}
\frametitle{Danke für Ihr Feedback!}

\begin{columns}[c]

\begin{column}{6cm}
\begin{center}
 \includegraphics[width=\textwidth]{smilie_balloons.jpg}
\end{center}

\end{column}

\begin{column}{4cm}


\begin{center}
\includegraphics[width=\textwidth]{feedback_QR.png}
\end{center}
\end{column}


\end{columns}

\end{frame}



%% %% %% Bildnachweis
\begin{frame}
\frametitle{Bildnachweis}
\begin{tiny}

Teile dieser Vorlesung wurden übernommen von einer Vorlesung von Prof. Emanuel Busch,  Health and Medical University Potsdam, dem wir an dieser Stelle herzlich danken. Wo nicht anders gekennzeichnet, stammen Abbildungen aus dieser Vorlesung.  


 
\begin{itemize}
\item
Assoziativität und Input-Spezifität von  LTP. Modifiziert von einer Illustration auf Quora (\url{https://www.quora.com/What-is-neuroplasticity-and-how-does-it-work}) - \emph{Bitte melden Sie sich, falls Sie die Originalquelle kennen}

\item
Berlin. Photo by \href{https://unsplash.com/@jonastebbe?utm_source=unsplash&utm_medium=referral&utm_content=creditCopyText}{Jonas Tebbe} on \href{https://unsplash.com/s/photos/berlin?utm_source=unsplash&utm_medium=referral&utm_content=creditCopyText}{Unsplash}

\item
Gestickte Versionen von Neuronen-Zeichnungen von Santiago Ramon y Cajal. From: A.R. Mehta, C.M. Abbott, S. Chandran, and J.E. Haley (2020). The Cajal Embroidery Project: celebrating neuroscience. \emph{The Lancet Neurology}, 19(12), p.979.

\item
Kind auf einem Fahhrad. Photo by \href{https://unsplash.com/@kornyeimiklos?utm_source=unsplash&utm_medium=referral&utm_content=creditCopyText}{Miklós Környei} on \href{https://unsplash.com/s/photos/child-cycling?utm_source=unsplash&utm_medium=referral&utm_content=creditCopyText}{Unsplash}
  
\item
Klassische Konditionierung. By Salehi.s - Own work, CC BY-SA 4.0, \url{https://commons.wikimedia.org/w/index.php?curid=53458610}

\item
Leipziger Platz. H.Helmlechner, CC BY-SA 4.0 \url{https://creativecommons.org/licenses/by-sa/4.0}, via Wikimedia Commons

\item
Long-term potentiation. By Synaptidude at English Wikipedia - This image created by Synaptidude 23:45, 15 August 2005 (UTC) , assembled in CorelDraw!, CC BY-SA 3.0, \url{https://commons.wikimedia.org/w/index.php?curid=886855}


%% all lectures
\item
Luftballons mit frohen und traurigen Smilies. Photo by \href{https://unsplash.com/@artbyhybrid?utm_source=unsplash&utm_medium=referral&utm_content=creditCopyText}{Hybrid} on \href{https://unsplash.com/s/photos/feedback?utm_source=unsplash&utm_medium=referral&utm_content=creditCopyText}{Unsplash}
%%%%%%%%%%%

\item
Ratte im Morris Water Maze. Jean-Etienne Minh-Duy Poirrier, CC BY-SA 2.0 \url{https://creativecommons.org/licenses/by-sa/2.0}, via Wikimedia Commons

\end{itemize}
\end{tiny}
\end{frame}






\end{document}

%%% Frequently used snippets

%% \begin{columns}[c]

%% \begin{column}{5cm}
%% \end{column}

%% \begin{column}{5cm}
%% \end{column}


%% \end{columns}




