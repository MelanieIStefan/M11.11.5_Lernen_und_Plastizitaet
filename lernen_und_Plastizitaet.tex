\documentclass{beamer}	
\mode<presentation>
 
\usepackage{pdfpages}
\usepackage{fancyvrb}
\usepackage{chemarr}

\usepackage{amsmath}		%% mathematics typesetting
\usepackage{amssymb}
 
\usepackage{epigraph}   %% nice setting of quotations

\usepackage{tabularx} %% allows to use row colours in tables

\usepackage{ulem}

\usepackage{booktabs}

\usepackage{siunitx} %% tpyeset SI units

\usepackage{CJKutf8} %% typeset Chinese characters

\usepackage{pdfpages}%% include pdfs

\usepackage{graphicx}
\usepackage{animate} %% show animated gifs

\DeclareMathAlphabet{\mathcalligra}{T1}{calligra}{m}{n}


% Color and Theme. Can be changed. However, this one's quite nice.
\usetheme{Madrid}
\definecolor{theme}{rgb}{0.84,0,0.21}
\usecolortheme[named=theme]{structure}


%%  Title information
\title[M11.11.5 Lernen, Gdächtnis]{M11.11.5 Zentrales Nervensystem 3: \\ Lernen und Gedächtnis}
\author[melanie.stefan@medicalschool-berlin.de]{}
\institute[]{Prof. Melanie Stefan \\ melanie.stefan@medcialschool-berlin.de}
\date{SoSe 2022}
 

% Table of contents to pop up at the beginning of each section
\AtBeginSection[]
{
  \begin{frame}<beamer>
    \frametitle{Outline}
    \tableofcontents[currentsection,currentsubsection]
  \end{frame}
}
 
\beamertemplatenavigationsymbolsempty

\begin{document}


{ \usebackgroundtemplate{\includegraphics[width=1.2\paperwidth]{MSB_Titelseite.pdf}} 
\begin{frame}

 \maketitle 

$\,$\\[6cm] 


\end{frame} 
}


%% Hook + TLIA

%% Water maze?
\begin{frame}
  \frametitle{Was passiert hier?}

\begin{center}
\includegraphics[width=0.6\textwidth]{MorrisWaterMaze.jpg}
\end{center}
  
\pause

Und was passiert derweil im Gehirn?
  
\end{frame}


 
%% %% TLIA



%% %% Learning Objectives
 
\begin{frame}

 \frametitle{Nach dieser Vorlesung sollten Sie folgendes können}



\begin{block}{Grundlagen:}
\begin{itemize}
\item
assoziative und nicht-assoziative Formen des Lernens beschreiben
\item
operante und klassische Konditionierung unterscheiden
\item
verschiedene Gedächtnisformen voneinander unterscheiden
\item
erklären, welche Gehirnstrukturen dem Gedächtnis zugrunde liegen
\item
Aufmerksamkeit definieren und die zugrundeliegenden Vorgänge im Gehirn beschreiben
\item
erklären, was eine Hebbsche Synapse ist
\item
die Eigenschaften von Langzeitpotenzierung (LTP) erklären
\item
die zellulären und molekularen Prozesse bei LTP erläutern
\end{itemize}

\end{block}



\end{frame}


 
\begin{frame}

 \frametitle{Nach dieser Vorlesung sollten Sie folgendes können}

\begin{block}{Klinik:}
\begin{itemize}
\item
Formen von Amnesie aufzählen und erläutern
\item
Neglect erklären und typische Symptome beschreiben
\item
häufig verwendete Tests für Lernen und Gedächtnis erläutern


\end{itemize}

\end{block}



\end{frame}










%% %% %% Main Body
 
 
\section{Aufmerksamkeit}
 
 \section{Gedächtnis}

\begin{frame}{Was ist Gedächtnis?}

Gedächtnis ist die Einheit von Merkfähigkeit und Erinnerung. 

(Information muss gespeichert und wieder aufgerufen werden). 

\pause

Dies geschieht durch die Bildung von Engrammen ("Gedächtnisspuren") zur Speicherung von Wahrnehmungen und Erfahrungen.​

\end{frame}

%% Engramme müssen niciht perfekt sein

%% Emanuel Folien zu Gedächtnisformen





 \section{Lernen}
 
\section{Synaptische Plastizität}
 


%% %% %% %% Review

\begin{frame}

 \frametitle{Jetzt* sollten Sie folgendes können}



\begin{block}{Grundlagen:}
\begin{itemize}
\item
assoziative und nicht-assoziative Formen des Lernens beschreiben
\item
operante und klassische Konditionierung unterscheiden
\item
verschiedene Gedächtnisformen voneinander unterscheiden
\item
erklären, welche Gehirnstrukturen dem Gedächtnis zugrunde liegen
\item
Aufmerksamkeit definieren und die zugrundeliegenden Vorgänge im Gehirn beschreiben
\item
erklären, was eine Hebbsche Synapse ist
\item
die Eigenschaften von Langzeitpotenzierung (LTP) erklären
\item
die zellulären und molekularen Prozesse bei LTP erläutern
\end{itemize}

\end{block}



\end{frame}


 
\begin{frame}

 \frametitle{Jetzt* sollten Sie folgendes können}

\begin{block}{Klinik:}
\begin{itemize}
\item
Formen von Amnesie aufzählen und erläutern
\item
Neglect erklären und typische Symptome beschreiben
\item
häufig verwendete Tests für Lernen und Gedächtnis erläutern


\end{itemize}

\end{block}



\end{frame}




%% %% %% %% Feedbackhinweisblock

\begin{frame}
\frametitle{Danke für Ihr Feedback!}

\begin{columns}[c]

\begin{column}{6cm}
\begin{center}
 \includegraphics[width=\textwidth]{smilie_balloons.jpg}
\end{center}

\end{column}

\begin{column}{4cm}


\begin{center}
\includegraphics[width=\textwidth]{feedback_QR.png}
\end{center}
\end{column}


\end{columns}

\end{frame}



%% %% %% Bildnachweis
\begin{frame}
\frametitle{Bildnachweis}
\begin{tiny}

Teile dieser Vorlesung wurden übernommen von einer Vorlesung von Prof. Emanuel Busch,  Health and Medical University Potsdam, dem wir an dieser Stelle herzlich danken. Wo nicht anders gekennzeichnet, stammen Abbildungen aus dieser Vorlesung.  


 
\begin{itemize}

%% all lectures
\item
Luftballons mit frohen und traurigen Smilies. Photo by \href{https://unsplash.com/@artbyhybrid?utm_source=unsplash&utm_medium=referral&utm_content=creditCopyText}{Hybrid} on \href{https://unsplash.com/s/photos/feedback?utm_source=unsplash&utm_medium=referral&utm_content=creditCopyText}{Unsplash}
%%%%%%%%%%%

\item
Ratte im Morris Water Maze. Jean-Etienne Minh-Duy Poirrier, CC BY-SA 2.0 \url{https://creativecommons.org/licenses/by-sa/2.0}, via Wikimedia Commons

\end{itemize}
\end{tiny}
\end{frame}






\end{document}

%%% Frequently used snippets

%% \begin{columns}[c]

%% \begin{column}{5cm}
%% \end{column}

%% \begin{column}{5cm}
%% \end{column}


%% \end{columns}




